\documentclass{article}
\usepackage{amsmath}
\usepackage{amssymb}
\title{README}
\author{Jeff Zhang}
\date{\today}
\begin{document}
\maketitle

Language models are probability distributions over sequences of tokens. More formally, we define the sample space ${\Omega}^d$ to be the d-dimensional cartesian product of a set of tokens, the event space $\mathcal{F}$ to be the set of all possible token sentences, and probability measure $\mathbb{P}: \mathcal{F} \mapsto [0,1].$

Letting the random vector $\mathbf{X}: {\Omega}^d \mapsto \mathbb{R}^d $, we can encode sentences into $\mathbb{R}^d$, which is a space that is computable (denotational vs distributional semantics?). To evaluate the probability of an encoded sentence, we compute:

\begin{equation}
p_\mathbf{X}(\mathbf{x}) = \mathbb{P}[\mathbf{X}=\mathbf{x}] = \mathbb{P}[\mathbf{X^{-1}}(\mathbf{x})] = \mathbb{P}[\{\omega \in {\Omega}^d: = \mathbf{X}(\omega) = \mathbf{x} \}]
\end{equation}

(autoregressive? markov assumption?)

The goal is to approximate $p_\mathbf{X}(\mathbf{x})$ given $\mathcal{D}={(x_1,y_1),...,(x_n,y_n)}$ (assuming $(x_i,y_i)~p$), with some hypothesis $h: \mathcal{X} \mapsto \mathcal {Y}$. With parametric models, this is done by posing parameter estimation as an optimization problem (ERM?) $\operatorname*{argmin}_\theta \mathcal{L}(\theta)$.

(PAC?, theoretical optimal?)






\end{document}